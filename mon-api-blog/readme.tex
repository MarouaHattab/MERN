\documentclass[12pt,a4paper]{article}
\usepackage[utf8]{inputenc}
\usepackage[french]{babel}
\usepackage{amsmath}
\usepackage{amsfonts}
\usepackage{amssymb}
\usepackage{graphicx}
\usepackage{xcolor}
\usepackage{listings}
\usepackage{hyperref}
\usepackage{geometry}
\usepackage{booktabs}
\usepackage{enumitem}
\usepackage{fancyhdr}
\usepackage{titlesec}

% Configuration de la page
\geometry{margin=2.5cm}

% Configuration des couleurs
\definecolor{codegreen}{rgb}{0,0.6,0}
\definecolor{codegray}{rgb}{0.5,0.5,0.5}
\definecolor{codepurple}{rgb}{0.58,0,0.82}
\definecolor{backcolour}{rgb}{0.95,0.95,0.92}

% Configuration du style de code
\lstdefinestyle{mystyle}{
    backgroundcolor=\color{backcolour},   
    commentstyle=\color{codegreen},
    keywordstyle=\color{magenta},
    numberstyle=\tiny\color{codegray},
    stringstyle=\color{codepurple},
    basicstyle=\ttfamily\footnotesize,
    breakatwhitespace=false,         
    breaklines=true,                 
    captionpos=b,                    
    keepspaces=true,                 
    numbers=left,                    
    numbersep=5pt,                  
    showspaces=false,                
    showstringspaces=false,
    showtabs=false,                  
    tabsize=2
}

\lstset{style=mystyle}

% Configuration des en-têtes
\pagestyle{fancy}
\fancyhf{}
\fancyhead[L]{Mon API Blog - Backend Express.js}
\fancyhead[R]{\thepage}

% Configuration des titres
\titleformat{\section}
{\Large\bfseries\color{blue}}{\thesection}{1em}{}

\titleformat{\subsection}
{\large\bfseries\color{darkblue}}{\thesubsection}{1em}{}

\title{\Huge\textbf{Mon API Blog - Backend Express.js}}
\author{Projet MERN - Semaine 1}
\date{\today}

\begin{document}

\maketitle
\tableofcontents
\newpage

\section{Description du Projet}

API RESTful développée avec Node.js et Express.js dans le cadre du cours MERN - Semaine 1. Ce projet pose les fondations d'une application back-end scalable suivant le principe de Séparation des Préoccupations (SoC).

\section{Objectifs Pédagogiques}

\begin{itemize}
    \item Concevoir une architecture back-end scalable
    \item Maîtriser le cycle de vie d'un projet NPM
    \item Construire un serveur Express avec routes GET et POST
    \item Comprendre le rôle des middlewares (express.json())
    \item Valider des endpoints d'API avec Postman
    \item Optimiser le flux de travail avec Nodemon
\end{itemize}

\section{Technologies Utilisées}

\begin{itemize}
    \item \textbf{Node.js} (Version LTS) - Environnement d'exécution JavaScript
    \item \textbf{Express.js} - Framework web minimaliste et flexible
    \item \textbf{Nodemon} - Outil de développement pour auto-reload
    \item \textbf{Postman} - Test et documentation d'API
\end{itemize}

\section{Structure du Projet (Vision Cible)}

\begin{lstlisting}[language=bash, caption=Structure des dossiers]
mon-api-blog/
├── node_modules/       # Dépendances installées par npm
├── config/             # Fichiers de configuration (ex: connexion BDD)
├── controllers/        # Logique métier
├── models/             # Schémas de données
├── routes/             # Définition des endpoints
├── .env                # Variables d'environnement
├── .gitignore          # Fichiers à ignorer par Git
├── package.json        # Manifeste du projet
└── server.js           # Point d'entrée de l'application
\end{lstlisting}

\section{Installation et Configuration}

\subsection{Prérequis}

Installer Node.js (Version LTS) et vérifier l'installation :

\begin{lstlisting}[language=bash]
node -v
npm -v
\end{lstlisting}

\subsection{Initialisation du Projet}

\begin{lstlisting}[language=bash]
mkdir mon-api-blog
cd mon-api-blog
npm init -y
\end{lstlisting}

\subsection{Installation des Dépendances}

\begin{lstlisting}[language=bash]
# Express - Framework pour créer le serveur et gérer les routes
npm install express

# Nodemon - Relance automatiquement le serveur à chaque modification
npm install nodemon --save-dev
\end{lstlisting}

\subsection{Configuration des Scripts NPM}

Modifier \texttt{package.json} :

\begin{lstlisting}[language=json]
"scripts": {
  "start": "node server.js",
  "dev": "nodemon server.js"
}
\end{lstlisting}

\textbf{Pourquoi ces scripts ?}
\begin{itemize}
    \item \texttt{start} : Mode production (sans auto-reload)
    \item \texttt{dev} : Mode développement avec Nodemon
\end{itemize}

\section{Code Complet du Serveur}

Créer le fichier \texttt{server.js} à la racine :

\begin{lstlisting}[language=javascript, caption=Code du serveur Express]
// --- Importation du module Express ---
const express = require('express');

// --- Création de l'application Express ---
const app = express();

// --- Définition du port d'écoute ---
const PORT = 3000;

// ============================================
// MIDDLEWARE
// ============================================

// --- Middleware pour parser le JSON ---
// Ce middleware permet de lire le corps (body) des requêtes POST/PUT au format JSON
// Il transforme le JSON reçu en objet JavaScript accessible via req.body
// IMPORTANT : Doit être placé AVANT la définition des routes POST
app.use(express.json());

// ============================================
// ROUTES GET
// ============================================

// --- Route racine (Page d'accueil) ---
// URL : http://localhost:3000/
// Méthode : GET
// Réponse : HTML simple
app.get('/', (req, res) => {
    res.status(200).send('<h1>Page d\'accueil de notre API de Blog !</h1>');
});

// --- Route de test de l'API ---
// URL : http://localhost:3000/api/test
// Méthode : GET
// Réponse : JSON avec message de confirmation
app.get('/api/test', (req, res) => {
    res.status(200).json({ 
        message: 'Le test a fonctionné !', 
        success: true 
    });
});

// --- Route "À propos" ---
// URL : http://localhost:3000/about
// Méthode : GET
// Réponse : Informations sur l'API au format JSON
app.get('/about', (req, res) => {
    res.status(200).json({
        app: 'API de blog',
        version: '1.0.0',
        description: 'API simple pour Atelier MERN',
    });
});

// --- Route pour récupérer les utilisateurs ---
// URL : http://localhost:3000/api/users
// Méthode : GET
// Réponse : Liste d'utilisateurs factices au format JSON
app.get('/api/users', (req, res) => {
    // Tableau d'utilisateurs fictifs (simule une base de données)
    const users = [
        { id: 1, nom: "Maroua", email: "maroua@gmail.com" },
        { id: 2, nom: "Sarra", email: "sarra@gmail.com" },
        { id: 3, nom: "Ahmed", email: "ahmed@gmail.com" }
    ];
    
    // Envoi de la réponse avec statut 200 (OK)
    res.status(200).json({ 
        count: users.length,  // Nombre d'utilisateurs
        users: users 
    });
});

// ============================================
// ROUTES POST
// ============================================

// --- Route pour créer un article ---
// URL : http://localhost:3000/api/articles
// Méthode : POST
// Body attendu : { "title": "...", "content": "...", "author": "..." }
// Réponse : Article créé avec un ID généré
app.post('/api/articles', (req, res) => {
    // Récupération des données envoyées dans le corps de la requête
    const articleData = req.body;
    
    // Affichage dans la console du serveur (utile pour le débogage)
    console.log('Données reçues :', articleData);
    
    // Simulation de la création d'un article avec un ID unique basé sur le timestamp
    res.status(201).json({
        message: 'Article créé avec succès !',
        article: { 
            id: Date.now(),  // Génère un ID unique basé sur le temps actuel
            ...articleData   // Spread operator : copie toutes les propriétés de articleData
        }
    });
});

// ============================================
// ROUTE CONTACT - VERSION NORMALE
// ============================================

// --- Route contact (Version Simple) ---
// URL : http://localhost:3000/contact
// Méthode : POST
// Body attendu : { "email": "...", "message": "..." }
app.post('/contact', (req, res) => {
    // Récupération des données du formulaire de contact
    const contactData = req.body;
    const email = contactData.email;
    const message = contactData.message;
    
    // Envoi de la réponse de confirmation
    res.status(200).json({
        message: `Message reçu de ${email} : ${message}`
    });
});

// ============================================
// DÉMARRAGE DU SERVEUR
// ============================================

// --- Lancement du serveur sur le port défini ---
app.listen(PORT, () => {
    console.log(`Serveur démarré sur http://localhost:${PORT}`);  
});
\end{lstlisting}

\section{Lancement du Serveur}

\subsection{Mode Développement (avec auto-reload)}

\begin{lstlisting}[language=bash]
npm run dev
\end{lstlisting}

\textbf{Sortie attendue :}
\begin{lstlisting}[language=bash]
Serveur démarré sur http://localhost:3000
\end{lstlisting}

\subsection{Mode Production}

\begin{lstlisting}[language=bash]
npm start
\end{lstlisting}

\section{Test des Endpoints avec Postman}

\subsection{Routes GET}

\subsubsection{Route Racine \texttt{/}}
\begin{itemize}
    \item \textbf{URL} : \texttt{http://localhost:3000/}
    \item \textbf{Méthode} : GET
    \item \textbf{Réponse} : HTML (page d'accueil)
\end{itemize}

\begin{lstlisting}[language=html]
<h1>Page d'accueil de notre API de Blog !</h1>
\end{lstlisting}

\subsubsection{Route Test \texttt{/api/test}}
\begin{itemize}
    \item \textbf{URL} : \texttt{http://localhost:3000/api/test}
    \item \textbf{Méthode} : GET
    \item \textbf{Réponse} : JSON
\end{itemize}

\begin{lstlisting}[language=json]
{
  "message": "Le test a fonctionné !",
  "success": true
}
\end{lstlisting}

\subsubsection{Route À Propos \texttt{/about}}
\begin{itemize}
    \item \textbf{URL} : \texttt{http://localhost:3000/about}
    \item \textbf{Méthode} : GET
    \item \textbf{Réponse} : JSON
\end{itemize}

\begin{lstlisting}[language=json]
{
  "app": "API de blog",
  "version": "1.0.0",
  "description": "API simple pour Atelier MERN"
}
\end{lstlisting}

\subsubsection{Route Utilisateurs \texttt{/api/users}}
\begin{itemize}
    \item \textbf{URL} : \texttt{http://localhost:3000/api/users}
    \item \textbf{Méthode} : GET
    \item \textbf{Réponse} : JSON
\end{itemize}

\begin{lstlisting}[language=json]
{
  "count": 3,
  "users": [
    { "id": 1, "nom": "Maroua", "email": "maroua@gmail.com" },
    { "id": 2, "nom": "Sarra", "email": "sarra@gmail.com" },
    { "id": 3, "nom": "Ahmed", "email": "ahmed@gmail.com" }
  ]
}
\end{lstlisting}

\subsection{Routes POST}

\subsubsection{Créer un Article \texttt{/api/articles}}
\begin{itemize}
    \item \textbf{URL} : \texttt{http://localhost:3000/api/articles}
    \item \textbf{Méthode} : POST
    \item \textbf{Headers} : \texttt{Content-Type: application/json}
    \item \textbf{Body (raw JSON)} :
\end{itemize}

\begin{lstlisting}[language=json]
{
  "title": "Mon premier article",
  "content": "Ceci est le contenu de mon article."
}
\end{lstlisting}

\textbf{Réponse attendue} : Status 201 Created

\begin{lstlisting}[language=json]
{
  "message": "Article créé avec succès !",
  "article": {
    "id": 1759182658631,
    "title": "Mon premier article",
    "content": "Ceci est le contenu de mon article."
  }
}
\end{lstlisting}

\subsubsection{Envoyer un Message de Contact \texttt{/contact}}
\begin{itemize}
    \item \textbf{URL} : \texttt{http://localhost:3000/contact}
    \item \textbf{Méthode} : POST
    \item \textbf{Headers} : \texttt{Content-Type: application/json}
    \item \textbf{Body (raw JSON)} :
\end{itemize}

\begin{lstlisting}[language=json]
{
  "email": "test@example.com",
  "message": "Bonjour, ceci est un message de test"
}
\end{lstlisting}

\textbf{Réponse attendue} : Status 200 OK

\begin{lstlisting}[language=json]
{
  "message": "Message reçu de test@example.com : Bonjour, ceci est un message de test"
}
\end{lstlisting}

\section{Codes Status HTTP Utilisés}

\begin{table}[h!]
\centering
\begin{tabular}{@{}lll@{}}
\toprule
\textbf{Code} & \textbf{Signification} & \textbf{Utilisation dans le projet} \\
\midrule
200 & OK & Requête GET réussie, message de contact reçu \\
201 & Created & Article créé avec succès \\
400 & Bad Request & Données manquantes ou invalides \\
404 & Not Found & Route inexistante \\
500 & Internal Server Error & Erreur serveur \\
\bottomrule
\end{tabular}
\caption{Codes de statut HTTP utilisés}
\end{table}

\section{Concepts Clés Expliqués}

\subsection{Express.js}
Framework qui simplifie la création de serveurs HTTP et la gestion des routes. Alternative élégante au module \texttt{http} natif de Node.js.

\textbf{Pourquoi Express ?}
\begin{itemize}
    \item Syntaxe simple et lisible
    \item Système de routing puissant
    \item Support des middlewares
    \item Large écosystème de plugins
\end{itemize}

\subsection{Middleware \texttt{express.json()}}
Permet de parser automatiquement le corps des requêtes JSON et de les rendre accessibles via \texttt{req.body}.

\textbf{Comment ça marche ?}
\begin{enumerate}
    \item Client envoie : \texttt{\{ "title": "Test" \}}
    \item Express reçoit des bytes bruts
    \item \texttt{express.json()} convertit en objet JS
    \item Accessible via : \texttt{req.body.title}
\end{enumerate}

\textbf{IMPORTANT} : Doit être déclaré avant les routes POST/PUT

\subsection{HTML vs JSON}

\begin{table}[h!]
\centering
\begin{tabular}{@{}lll@{}}
\toprule
\textbf{Aspect} & \textbf{HTML} & \textbf{JSON} \\
\midrule
Usage & Affichage dans navigateurs & Échange de données APIs \\
Format & Langage de balisage & Format de données \\
Exemple & \texttt{<h1>Titre</h1>} & \texttt{\{ "title": "Titre" \}} \\
Destiné à & Humains (visuel) & Machines (traitement) \\
Content-Type & \texttt{text/html} & \texttt{application/json} \\
\bottomrule
\end{tabular}
\caption{Comparaison HTML vs JSON}
\end{table}

\subsection{Nodemon}
Outil de développement qui surveille les modifications de fichiers et relance automatiquement le serveur.

\textbf{Avantages :}
\begin{itemize}
    \item ✓ Gain de temps considérable
    \item ✓ Pas de redémarrage manuel
    \item ✓ Détection automatique des changements
    \item ✓ Configuration simple
\end{itemize}

\section{Travail Pratique Réalisé}

\subsection{Tâches Accomplies}

\begin{itemize}
    \item ✓ \textbf{Route "À Propos"} : \texttt{GET /about} - Retourne les infos de l'API
    \item ✓ \textbf{Route Utilisateurs} : \texttt{GET /api/users} - Liste d'utilisateurs factices
    \item ✓ \textbf{Route Contact} : \texttt{POST /contact} - Gestion des messages de contact
    \item ✓ \textbf{Tests Postman} : Validation de toutes les routes
\end{itemize}

\subsection{Compétences Acquises}

\begin{itemize}
    \item Configuration d'un projet Node.js avec NPM
    \item Création d'un serveur Express
    \item Gestion des routes GET et POST
    \item Utilisation de middlewares
    \item Parsing de données JSON
    \item Test d'API avec Postman
    \item Validation de données (version améliorée)
    \item Gestion d'erreurs HTTP
\end{itemize}

\section{Ressources Utiles}

\begin{itemize}
    \item Documentation Express.js : \url{https://expressjs.com/}
    \item Documentation Node.js : \url{https://nodejs.org/docs/}
    \item Guide Postman : \url{https://learning.postman.com/}
    \item MDN - HTTP Status Codes : \url{https://developer.mozilla.org/fr/docs/Web/HTTP/Status}
    \item NPM Documentation : \url{https://docs.npmjs.com/}
\end{itemize}

\section{Licence}

Ce projet est à but éducatif dans le cadre du cours MERN de l'École Polytechnique de Sousse.

\end{document}